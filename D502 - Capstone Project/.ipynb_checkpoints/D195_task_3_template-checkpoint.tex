Title of Capstone Here

Author's Full Name Here, Including Middle Initial

Western Governors University

\textbf{\hfill\break
}

\hypertarget{table-of-contents}{%
\section{Table of Contents}\label{table-of-contents}}

\protect\hyperlink{a.-project-highlights}{A. Project Highlights 4}

\protect\hyperlink{b.-project-execution}{B. Project Execution 5}

\protect\hyperlink{c.-data-collection-process}{C. Data Collection
Process 5}

\protect\hyperlink{c.1-advantages-and-limitations-of-data-set}{C.1
Advantages and Limitations of Data Set 6}

\protect\hyperlink{d.-data-extraction-and-preparation}{D. Data
Extraction and Preparation 6}

\protect\hyperlink{e.-data-analysis-process}{E. Data Analysis Process 6}

\protect\hyperlink{e.1-data-analysis-methods}{E.1 Data Analysis Methods
6}

\protect\hyperlink{e.2-advantages-and-limitations-of-tools-and-techniques}{E.2
Advantages and Limitations of Tools and Techniques 7}

\protect\hyperlink{e.3-application-of-analytical-methods}{E.3
Application of Analytical Methods 7}

\protect\hyperlink{f-data-analysis-results}{F Data Analysis Results 7}

\protect\hyperlink{f.1-statistical-significance}{F.1 Statistical
Significance 7}

\protect\hyperlink{f.2-practical-significance}{F.2 Practical
Significance 8}

\protect\hyperlink{f.3-overall-success}{F.3 Overall Success 8}

\protect\hyperlink{g.-conclusion}{G. Conclusion 9}

\protect\hyperlink{g.1-summary-of-conclusions}{G.1 Summary of
Conclusions 9}

\protect\hyperlink{g.2-effective-storytelling}{G.2 Effective
Storytelling 9}

\protect\hyperlink{g.3-recommended-courses-of-action}{G.3 Recommended
Courses of Action 9}

\protect\hyperlink{h-panopto-presentation}{H Panopto Presentation 9}

\protect\hyperlink{references}{References 12}

\protect\hyperlink{appendix-a}{Appendix A 12}

\protect\hyperlink{title-of-appendix}{Title of Appendix 13}

\protect\hyperlink{appendix-b}{Appendix B 13}

\protect\hyperlink{title-of-appendix-1}{Title of Appendix 14}

\protect\hyperlink{appendix-c}{Appendix C 14}

\protect\hyperlink{title-of-appendix-2}{Title of Appendix 15}

\protect\hyperlink{appendix-d}{Appendix D 15}

\protect\hyperlink{title-of-appendix-3}{Title of Appendix 16}

Task 3 is a report of the concluded project from Task 2. If you followed
our recommendation of completing your analysis before writing Task 2,
then most of Task 3 will simply repeat and present what was completed in
Task 2, and it can be completed very quickly. However, variances between
Task 2 and Task 3 are allowed (and should be discussed in part B), and
Task 3 is assessed independently of Task 2. So misalignment between the
two is acceptable, provided that Task 3 meets all its rubric
requirements. Rubric part I is met by submitting all code and data (or
link(s) to the data) so is not included in this template.

Note: this template is \emph{not official} and in development. Feedback
is appreciated!
\href{mailto:jim.ashe@wgu.edu}{\nolinkurl{jim.ashe@wgu.edu}}

\begin{itemize}
\item
  \textbf{Bold Section Headers =} part of the template. Do not remove.
\item
  Black text = general recommendations (remove before submitting).
\item
  Green text = official rubric requirements for competency (remove
  before submitting).
\item
  Red text = unofficial recommendations warranting special attention
  (remove before submitting).
\end{itemize}

Follow APA guidelines and check grammar. Task 3 is a \emph{conclusion}
report. So the narrative tense and dates should be in the past; be
particularly careful when reusing content from Task 2. You are allowed
and encouraged to reuse any of your work as needed. The similarity check
will not check for plagiarism against your content.

\hypertarget{a.-project-highlights}{%
\section{A. Project Highlights}\label{a.-project-highlights}}

\textbf{Rubric A:} The summary includes each of the given elements, and
the summary of each element is detailed. All the information in the
summary is accurate.

The ``given'' elements to summarize are provided in the task directions
as follows:

\begin{itemize}
\item
  The research question or organizational need that your capstone
  addressed (see task 2 section A1). If it needs to be revised from Task
  2, you can do so here.
\item
  The scope of your project (see task 2 section B2).
\item
  An overview of your solution, including any tools and methodologies
  used (see task 2 sections B3 and C3).
\end{itemize}

You can reuse content from Task 2 as needed. You are allowed and
encouraged to reuse any of your work. The similarity check will not
check for plagiarism against your content.

\hypertarget{b.-project-execution}{%
\section{B. Project Execution}\label{b.-project-execution}}

\textbf{Rubric B:} The summary accurately addresses how the execution
of~\emph{each}~element differed from the plan developed in part B of
Task 2.

This part summarizes the project's execution and any variances from the
execution plan provided in Task 2. ``Each'' element for which to
summarize variances are provided in the task directions as follows:

\begin{itemize}
\item
  The project plan (see Task 2 section B1).
\item
  The project planning methodology (see Task 2 section B3).
\item
  Project timeline and milestones (see Task 2 section B4).
\end{itemize}

If the element did not vary from Task 2, explain that there were no
variances and why. You can reuse content from Task 2 as needed. You are
allowed and encouraged to reuse any of your work. The similarity check
will not check for plagiarism against your content.

\hypertarget{c.-data-collection-process}{%
\section{C. Data Collection Process}\label{c.-data-collection-process}}

\textbf{Rubric C:} The discussion of the data selection and collection
process includes specific details for~\emph{each~}of the given elements.

This part summarizes the data selection, collection, processing, and
governance issues. The ``given'' elements to summarize are provided in
the task directions as follows:

\begin{itemize}
\item
  How the data selection and collection differed from your plan (see
  Task 2 section D3).
\item
  How you handled any obstacles encountered while collecting your data
  (see Task 2 section D4).
\item
  How you handled any unplanned data governance issues (see Task 2
  section D5).
\end{itemize}

If the element did not vary from Task 2 or you did not encounter any
unplanned issues, explain why. You can reuse content from Task 2 as
needed. You are allowed and encouraged to reuse any of your work. The
similarity check will not check for plagiarism against your content.

\hypertarget{c.1-advantages-and-limitations-of-data-set}{%
\subsection{C.1 Advantages and Limitations of Data
Set}\label{c.1-advantages-and-limitations-of-data-set}}

\textbf{Rubric C1:} The discussion addresses~\emph{both~}the advantages
and limitations of the data set, including specific examples
of~\emph{both}.~\emph{All}~discussed advantages and limitations apply to
the data set that was used.

\begin{itemize}
\item
  Discuss at least one advantage of your data set, and provide at least
  one example of an advantage.
\item
  Discuss at least one disadvantage of your data set, and provide at
  least one example of an advantage.
\end{itemize}

\hypertarget{d.-data-extraction-and-preparation}{%
\section{D. Data Extraction and
Preparation}\label{d.-data-extraction-and-preparation}}

\textbf{Rubric D:} The submission explains~\emph{both}~the data
extraction and data preparation processes, including details on why the
processes were appropriate for the data. The explanation includes the
tools and techniques that were used for~\emph{both~}processes.

Discuss the data extraction and preparation process and their
appropriateness (see Task 2 part D), and discuss the tools used for both
extractions. If little or no extraction or processing was necessary,
explain why.

\hypertarget{e.-data-analysis-process}{%
\section{E. Data Analysis Process}\label{e.-data-analysis-process}}

\hypertarget{e.1-data-analysis-methods}{%
\subsection{E.1 Data Analysis Methods}\label{e.1-data-analysis-methods}}

\textbf{Rubric E1:} The submission includes a thorough and accurate
description of data analysis methods that are appropriate for the
intended goals of the project.

Name and describe each method used to analyze the data. Include the
method(s) used to support your hypothesis in Task 3 part F (below; also
see Task 2 part C). Explain why each method was appropriate.

\hypertarget{e.2-advantages-and-limitations-of-tools-and-techniques}{%
\subsection{E.2 Advantages and Limitations of Tools and
Techniques}\label{e.2-advantages-and-limitations-of-tools-and-techniques}}

\textbf{Rubric E2:} The submission accurately discusses the advantages
and the limitations of the tools and techniques used to analyze the
data.

Discuss at least one advantage and one limitation of each tool used
during the analysis (see Task 2 section C3).

\hypertarget{e.3-application-of-analytical-methods}{%
\subsection{\texorpdfstring{E.3 Application of Analytical Methods
}{E.3 Application of Analytical Methods }}\label{e.3-application-of-analytical-methods}}

\textbf{Rubric E3:} The submission includes a thorough step-by-step
explanation of how the analytical methods were applied to the data and
how~\emph{all}~assumptions or requirements were verified.

Describe the steps used to complete each method used for the data
analysis in part F of Task 3 (see Task 2 section C2). Describe the
requirements for each method and how they were verified.

\hypertarget{f-data-analysis-results}{%
\section{\texorpdfstring{F Data Analysis Results
}{F Data Analysis Results }}\label{f-data-analysis-results}}

\hypertarget{f.1-statistical-significance}{%
\subsection{F.1 Statistical
Significance}\label{f.1-statistical-significance}}

\textbf{Rubric F1:} A thorough evaluation of the statistical
significance of the analysis is provided, and the evaluation uses
accurate calculations.

This section should report the results of the planned statistical
test(s) or model(s) from Task 2 section C4. For at least one analytic
method provide all the items for a \emph{statistical test} or a
\emph{model}:

For \emph{statistical tests}, provide the following information:~

\begin{itemize}
\item
  The null hypothesis (the opposite of your hypothesis).~
\item
  The name of the statistical test.~
\end{itemize}

\begin{itemize}
\item
  The metric(s) generated from that test (e.g., a t-stat and the derived
  probability).~
\item
  The \emph{alpha} value (denoted α; usually 1\% or 5\%) given in Task 2
  section C4.
\item
  The conclusion drawn, e.g., ``There is sufficient evidence to reject
  the null hypothesis and support the claim that (your hypothesis).''~~
\end{itemize}

For \emph{models}, provide the following information:~

\begin{itemize}
\item
  The type of model, e.g., supervised regression, supervised
  classification, etc.~~
\item
  The algorithm(s) and process used to develop the model.~
\end{itemize}

\begin{itemize}
\item
  The metric(s) used to assess performance.~~
\item
  The benchmark for the success of the above metric is given in Task 2
  section C4
\item
  The conclusion is drawn from that metric and how it supports or does
  not support your hypothesis.
\end{itemize}

\hypertarget{f.2-practical-significance}{%
\subsection{\texorpdfstring{F.2 Practical Significance
}{F.2 Practical Significance }}\label{f.2-practical-significance}}

\textbf{Rubric F2:} A thorough and accurate evaluation of the practical
significance of the data analytics solution is provided, and the
evaluation is supported by specific examples.

Discuss the practical significance of the results from F1. This can
repeat what you wrote in Task 2 section C5 adjusting as necessary
according to the results. Practical significance refers to how
meaningful your findings are in practical application. Results are
practically significant when the difference is large enough to be
meaningful in real life. This is subjective; so try your best to frame
the results as a success.

Include an example of how the client might apply your work discussed in
section F1.

\hypertarget{f.3-overall-success}{%
\subsection{F.3 Overall Success}\label{f.3-overall-success}}

\textbf{Rubric F3:} A thorough and accurate evaluation of the overall
success and effectiveness of the project is provided.

Based on the results presented in F1 and F2, discuss how the project was
successful. This section may repeat content from sections F1, and F2
above, and Task 2 section B6.

\hypertarget{g.-conclusion}{%
\section{G. Conclusion}\label{g.-conclusion}}

\hypertarget{g.1-summary-of-conclusions}{%
\subsection{G.1 Summary of
Conclusions}\label{g.1-summary-of-conclusions}}

\textbf{Rubric G1:} The conclusions drawn from the analysis are
summarized and are appropriate and logically consistent with the data
set, chosen analytic methods, and stated goals of the project.

Summarize your conclusions resulting from the entire project. This
section can combine, repeat, and expand on content from throughout Task
3.

\hypertarget{g.2-effective-storytelling}{%
\subsection{G.2 Effective
Storytelling}\label{g.2-effective-storytelling}}

\textbf{Rubric G2:} The explanation includes logical reasons why the
chosen tools and graphical representations for visually communicating
the findings support effective storytelling.

Summarize your visualizations and how they support effective
storytelling. Discuss all graphical representations and the tools used
for the development.

\hypertarget{g.3-recommended-courses-of-action}{%
\subsection{G.3 Recommended Courses of
Action}\label{g.3-recommended-courses-of-action}}

\textbf{Rubric G3:} The submission recommends~\emph{2~}courses of
action, and~\emph{both}~are logical, based on the findings, follow
logically from the analysis, and directly address the research question
or organizational need of the project.

Provide TWO recommendations based on the results (section F2) and
conclusion (section G1) of your data analysis. Explain how each
recommendation relates to the research question or organizational need
given in Task 2 section A1 or redefined in Task 3 part A.

\hypertarget{h-panopto-presentation}{%
\section{H Panopto Presentation}\label{h-panopto-presentation}}

\textbf{Rubric H:} A link to a Panopto recording is provided, and the
summary includes~\emph{each}~of the given elements. The summary is
appropriate for an audience of data analytics peers.

Provide a link to your Panopto video.

\begin{itemize}
\item
  You are required to use Panopto, and you need to first request access,
  \href{https://wgu.hosted.panopto.com/}{Panopto Access}. Gaining
  permission may take up to 48 hours.
\item
  Your recording should include your voice, but showing your face is not
  required. If you need special accommodations, please contact
  \href{mailto:assessmentservices@wgu.edu?subject=D195\%20task\%203\%20Panopto\%20Video}{Assessment
  Services} and alert your assigned course faculty.
\item
  There is no minimum length, but it is not meant to be long. Depending
  on your project, approximately 5-15 minutes is a good length.
\end{itemize}

The video should provide the viewer with a bird's-eye view of your
project and how you conducted your analysis. Here are the points you
should cover:

\begin{enumerate}
\def\labelenumi{\arabic{enumi}.}
\item
  A summary of your research question or organizational need.
\item
  A summary demonstration of the functionality of any code you used for
  your data analytics solution.
\item
  An outline of the findings and implications of your analysis.
\end{enumerate}

The video should summarize your question and findings. You can think of
this as a ``water cooler'' version of your report. The evaluator should
be able to watch the video and understand your project's purpose and
main argument. Then, for step 2, bring your code (or software analysis)
on screen and step through how you conducted your analysis.
Unfortunately, we don't have an example video. However, I never see the
video rejected unless it's missing step 2 above.\\
\hspace*{0.333em}

\hypertarget{references}{%
\section{References}\label{references}}

No sources were cited.

\hypertarget{appendix-a}{%
\section{Appendix A}\label{appendix-a}}

\hypertarget{title-of-appendix}{%
\section{Title of Appendix}\label{title-of-appendix}}

Put any supporting material in these appendices. Add additional or
delete superfluous appendices as needed.

\hypertarget{appendix-b}{%
\section{Appendix B}\label{appendix-b}}

\hypertarget{title-of-appendix-1}{%
\section{Title of Appendix}\label{title-of-appendix-1}}

Put any supporting material in these appendices. Add additional or
delete superfluous appendices as needed.

\hypertarget{appendix-c}{%
\section{Appendix C}\label{appendix-c}}

\hypertarget{title-of-appendix-2}{%
\section{Title of Appendix}\label{title-of-appendix-2}}

Put any supporting material in these appendices. Add additional or
delete superfluous appendices as needed.

\hypertarget{appendix-d}{%
\section{Appendix D}\label{appendix-d}}

\hypertarget{title-of-appendix-3}{%
\section{Title of Appendix}\label{title-of-appendix-3}}

Put any supporting material in these appendices. Add additional or
delete superfluous appendices as needed.
